%%%%%%%%%%%%%%%%%%%%%%%%%%%%%%%%%%%%%%%%%%%%%%%%%%%%%%%%%%%%%%%%%%%%%%%%%%%%%%%%%%%%%%%%%%%%%%%%%%%%%%%
%%%%%%%%%%%%%% Template de Artigo Adaptado para Trabalho de Diplomação do ICEI %%%%%%%%%%%%%%%%%%%%%%%%
%% codificação UTF-8 - Abntex - Latex -  							     %%
%% Autor:    Fábio Leandro Rodrigues Cordeiro  (fabioleandro@pucminas.br)                            %% 
%% Co-autores: Prof. João Paulo Domingos Silva, Harison da Silva e Anderson Carvalho		     %%
%% Revisores normas NBR (Padrão PUC Minas): Helenice Rego Cunha e Prof. Theldo Cruz                  %%
%% Versão: 1.1     18 de dezembro 2015                     %%
%%%%%%%%%%%%%%%%%%%%%%%%%%%%%%%%%%%%%%%%%%%%%%%%%%%%%%%%%%%%%%%%%%%%%%%%%%%%%%%%%%%%%%%%%%%%%%%%%%%%%%%
\section{Alfabeto $\Sigma$} 
\begin{table}[!h]
\centering
\caption{Elementos do alfabeto $\Sigma$}
\vspace{0.2cm}
\begin{tabular}{r}
 
Elemento \\ % Note a separação de col. e a quebra de linhas
\hline                               % para uma linha horizontal
\textit{\textbf{final}}\\
\textit{\textbf{else}}\\
\textit{\textbf{(}}\\
\textit{\textbf{<=}}\\
\textit{\textbf{;}}\\
\textit{\textbf{write}}\\
\textit{\textbf{int}}\\
\textit{\textbf{\&\&}}\\
\textit{\textbf{)}}\\
\textit{\textbf{,}}\\
\textit{\textbf{begin}}\\
\textit{\textbf{writeln}}\\
\textit{\textbf{byte}}\\
\textit{\textbf{||}}\\
\textit{\textbf{<}}\\
\textit{\textbf{+}}\\
\textit{\textbf{endwhile}}\\
\textit{\textbf{TRUE}}\\
\textit{\textbf{string}}\\
\textit{\textbf{!}}\\
\textit{\textbf{>}}\\
\textit{\textbf{-}}\\
\textit{\textbf{endif}}\\
\textit{\textbf{FALSE}}\\
\textit{\textbf{while}}\\
\textit{\textbf{<-}}\\
\textit{\textbf{!=}}\\
\textit{\textbf{*}}\\
\textit{\textbf{endelse}}\\
\textit{\textbf{boolean}}\\
\textit{\textbf{if}}\\
\textit{\textbf{=}}\\
\textit{\textbf{>=}}\\
\textit{\textbf{/}}\\
\textit{\textbf{readln}}\\
\end{tabular}
\end{table}


\section{\esp Lexemas e Padrão de formação}
\begin{table}[!h]
\centering
\caption{Lexema x Padrão de formação}
\vspace{0.2cm}
\begin{tabular}{r|lr}
 
Posi{\c c}{\~a}o & Lexema & Padr{\~a}o de Forma{\c c}{\~a}o \\ % Note a separação de col. e a quebra de linhas
\hline                               % para uma linha horizontal
1 & final & final\\
2 & else & else\\
3 & ( & ( \\
4 & <= & <= \\
5 & ; & ; \\
6 & write & write\\
7 & int & int\\
8 & \&\& & \&\& \\
9 & ) & ) \\
10 & , & , \\
11 & begin & begin\\
12 & writeln & writeln\\
13 & byte & byte\\
14 & || & || \\
15 & < & < \\
16 & + & + \\
17 & endwhile & endwhile\\
18 & TRUE & TRUE \\
19 & string & string\\
20 & ! & ! \\
21 & > & > \\
22 & - & - \\
23 & endif & endif\\
24 & FALSE & FALSE \\
25 & while & while\\
26 & <- & <- \\
27 & != & !=  \\
28 & * & * \\
29 & endelse & endelse\\
30 & boolean & boolean\\
31 & if & if\\
32 & = & = \\
33 & >= & >= \\
34 & / & / \\
35 & readln & readln\\
\end{tabular}
\end{table}


% \subsection{\esp Trabalhos futuros}
% 
% Sugestões de estudos posteriores são ser adicionados subseção deste capítulo de conclusão.
